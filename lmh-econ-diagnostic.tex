\documentclass[addpoints,12pt]{exam}
\usepackage{amsmath, amssymb}
\usepackage[T1]{fontenc}
\boxedpoints
\pointsinmargin

\usepackage{datetime2}
\usepackage{bbm}
\usepackage{tikz}
\usepackage{array}
\usepackage{multirow}
\usepackage{amsmath}
\usepackage{amsthm}
\usepackage{amssymb}
\usepackage[margin=1in]{geometry}
\usepackage{enumitem}
\usepackage{array}     
\usepackage{booktabs}
\usepackage{multirow}
\usepackage{array}
\usepackage{hyperref}
\hypersetup{colorlinks=true}

\renewcommand{\thesection}{\Alph{section}}


\begin{document}


\begin{center}

  \textsc{\Large Economics Diagnostic Test \\ \normalsize Lady Margaret Hall Visiting Student Program \\ \vspace{0.5em} Version: \DTMtoday, \DTMcurrenttime}

  \vspace{1em}

  \href{https://raw.githubusercontent.com/fditraglia/lmh-econ-diagnostic/main/lmh-econ-diagnostic.pdf}{\underline{Click here for latest version}}

\vspace{5em}

\fbox{\begin{minipage}{0.5\textwidth}
\normalsize\textbf{This test is open book: notes, books, and calculators are allowed; AI is not. You may not consult anyone for help. There is no time limit.}\end{minipage}}


\end{center}
%%%%%%%%%%%%%%%%%%%%%%%%%%%%%%%%%%%%%%%%%%%%%%%%%%%%%%%%%%%%%%%

\vspace{2em}

\paragraph{All visiting students applying to take courses in Economics at Lady Margaret Hall should complete this diagnostic test.}

\vspace{2em}

\paragraph{Instructions:} This is a diagnostic test to help us determine which economics courses at Oxford will be appropriate for you, given your background. Depending on the economics courses you have taken or are currently taking, you will be asked to answer a number of relevant questions. Each section below begins with an explanation of which applicants should attempt the questions in that section. For a summary of this information, see the table at the bottom of this page. Answer each question to the best of your ability. Some questions are short and others are longer, but even the very longest should take less than an hour. If you get stuck, or encounter a topic or concept with which you are unfamiliar, please note this down in lieu of a full answer and move on to attempt the remaining questions. As a rough guide, expect to spend between 45 and 90 minutes on each section that you are required to complete. This document may be updated from time to time, e.g.\ to fix typos. Please click on the link above to ensure you have the most recent version!


\vspace{2em}

\begin{center}
  \begin{tabular}[h]{cl}
  \textbf{Section} & \textbf{Attempt if you have taken:}\\
    \hline
    A & All applicants should complete this section.\\
    B & Introductory Prob/Stats \\
    C & Intermediate Micro \\
    D & Intermediate Micro \\
    E & Advanced Micro\\
    F & Intermediate Macro\\
    G & Advanced Macro\\
    H & Introductory Econometrics \\
  \end{tabular}
\end{center}




%%%%%%%%%%%%%%%%%%%%%%%%%%%%%%%%%%%%%%%%%%%%%%%%%%%%%%%%%%%%%%%
\newpage

%\begin{questions}

\section{Calculus for Economics}
\textbf{All applicants} should attempt the questions in this section. 

\begin{questions}

\question Solve the simultaneous equations $y = x^2 + 1$ and $2y - 3x = 4$.
\question Find all solutions of the equation $x^3 - 2x^2 - 9x + 18 = 0$.
\question Simplify $\frac{1}{2}\ln x^4 + 3\ln(2y)$.
\question Evaluate the sum of the infinite series $2 + 1 + 0.5 + 0.25 + 0.125 + \cdots$.
\question Find all stationary points of the function $f(x,y) = x^3 - y^3 - 3x + 12y$, and classify them.
\question Find the maximum and the minimum of the function $f(x,y) = 2x + y^2$ subject to the constraint $x + y = 4$, where $x$ and $y$ are both non-negative real numbers.


\fullwidth{\section{Introductory Probability \& Statistics}
Attempt these if you have taken or are currently taking \textbf{Introductory Probability \& Statistics}.}

\item Last year 24\% of economics majors graduated with an A average, 70\% graduated with a B average, and 6\% graduated with a C average. (No one who graduated averaged below a C.) Economics majors can choose to write a thesis in their senior year, and 40\% of majors did not write a thesis. Of the majors who did not write a thesis, only 15\% graduated with an A average. Consider a randomly chosen economics major who graduated with an A average. What is the probability that this student wrote a thesis?
\question Briefly answer each of the following:
\begin{parts}
  \part What does it mean for an estimator to be \emph{consistent}? Explain and give an example and counterexample.
  \part What does it mean for an estimator to be \emph{efficient}? Explain and give an example and counterexample.
\part Define the term \emph{mean independent}. Then explain why the error term $U$ in the identity $Y = \mathbbm{E}[Y|X] + U$ is mean independent of $X$.
\part Explain the term ``selection bias'' using the potential outcomes framework. 
\end{parts}


\newpage


\fullwidth{\section{Microeconomic Theory I}
Attempt these if you have taken or are currently taking an \textbf{Intermediate Microeconomics} or an \emph{advanced} Introduction to Microeconomics course.}

\question Briefly answer each of the following parts.
\begin{parts}
  \part Consider an asset that pays \$200 per year in perpetuity. If the interest rate is fixed at 4\% forever, what is the present value of this asset? 
  \part Firms operating in a competitive market make zero profit in the long run. Why do they still produce? 
  \part A monopolist faces demand $q(p)$. Find the elasticity of demand at the revenue-maximising quantity.
\end{parts}
  \question Arthur lives in a world in which there are just two goods: nutmegs and pears. In his garden there is a tree which yields eight nutmegs and one pear every day. He has no income other than from his nutmegs and pears. Arthur's preferences over nutmegs and pears may be represented by the utility function $u(n,p) = 2\ln n + 3\ln p$, where $n$ is the number of nutmegs that he consumes and $p$ is the number of pears.
  \begin{parts}
\part Show that Arthur's preferences may, alternatively, be represented by a utility function of the form $v(n,p) = n^\alpha p^\beta$ and explain why this is the case.
\part People in Arthur's world are prepared to trade one pear for two nutmegs. Describe the relationship between the prices of pears and nutmegs and draw a carefully labelled graph of Arthur's budget constraint.
\part Arthur maximises his utility subject to his budget constraint. Show that his gross demands are 4 nutmegs and 3 pears. Mark the gross demands on your diagram and sketch in one or two of his indifference curves. What are his net demands?
\part There is a shortage of nutmegs in Arthur's world, so the relative price of nutmegs increases. (Arthur's tree still produces the same yield every day.) Illustrate on your diagram what happens to his budget constraint. Will he be better off or worse off after the price change? What can you say about how his gross demands will change?
\part Consider the effect of the change in the price of nutmegs on Arthur's demand for nutmegs. This may be decomposed into a substitution effect, an ordinary income effect and an endowment income effect. Explain what is meant by these terms and draw a diagram to illustrate this decomposition. (Please draw a new graph for this part of the question.)
  \end{parts}


\fullwidth{\section{Introduction to Game Theory}
Attempt these if you have taken \textbf{Intermediate Microeconomics} or an \emph{advanced} Introduction to Microeconomics.}

\question Briefly answer each of the following parts. 
\begin{parts}
  \part What does it mean for a strategy to be dominant? If players choose dominant strategies will they necessarily play according to a Nash Equilibrium? If players play according to a Nash Equilibrium, will they necessarily play dominant strategies?
  \part For each of the following two-player, one-shot, simultaneous-move games, can you predict how the players will play? Why or why not? Explain briefly.
  \begin{itemize}
    \item[(i)] Each player decides between \emph{Green} and \emph{Red}. The payoffs are as follows: if both choose the same action, each receives 5; otherwise the player who chooses \emph{Green} receives 6 and the player who chose \emph{Red} receives 4.
    \item[(ii)] Each player decides between \emph{Blue} and \emph{Yellow}. The payoffs are as follows: if both choose \emph{Blue}, each receives 3; if both choose \emph{Yellow}, each receives 5; otherwise the player who chose \emph{Blue} receives 4 while the player who chose \emph{Yellow} receives 2.
  \end{itemize}
  \part Briefly discuss the usefulness of the concept of Nash Equilibrium for predicting the outcome of a game.
\end{parts}
\question Consider the following two-player, one-shot, simultaneous-move game:


% Define a new command for the bi-matrix cells with precise positioning
\newcommand{\cell}[2]{%
    \begin{tikzpicture}[baseline=(current bounding box.center)]
        % Create a precisely sized box
        \node[minimum width=2cm, minimum height=2cm, outer sep=0pt] (box) {};
        % Draw the outer box with precise corners
        \draw[line width=0.4pt] (box.north west) rectangle (box.south east);
        % Draw the diagonal with precise endpoints
        \draw[line width=0.4pt] (box.north west) -- (box.south east);
        % Position numbers with precise offsets
        \node[anchor=south west, xshift=8pt, yshift=8pt] at (box.south west) {#1};
        \node[anchor=north east, xshift=-8pt, yshift=-8pt] at (box.north east) {#2};
    \end{tikzpicture}%
}


\begin{table}[h!]
\centering
\begin{tabular}{cc|c|c|}
    \multicolumn{2}{c}{} & \multicolumn{2}{c}{Column Player} \\
    \multicolumn{2}{c}{} & Left & Right \\ \cline{3-4}
    \multirow{2}{*}{Row Player} & Up    & \cell{6}{4} & \cell{3}{5} \\ \cline{3-4}
    & Down  & \cell{5}{3} & \cell{2}{2} \\ \cline{3-4}
\end{tabular}
\end{table}

\begin{parts}
  \part This game has a unique pure-strategy Nash equilbrium. Identify it and explain your reasoning. 
  \fullwidth{Above we assumed that this was a simultaneous-move game. Now suppose that Row player moves first, followed by Column player.}
  \part Represent the game in extensive form and identify the subgame perfect equilibrium of the game. Is the subgame perfect equilibrium a Nash equilibrium? Explain briefly.
  \part How would your answer to the preceding part change if instead Column player moved first?
  \part Consider the two sequential games that you have analyzed in the preceding two parts. Does either have a Nash equilibrium that is not subgame perfect? If not, explain why. If so, identify it and explain why it is Nash but not subgame perfect.
\end{parts}

\question Consider an industry with two firms. Each has an identical cost function: $c(q_k) = 2q_k$ for $k = 1, 2$. The inverse demand function in this industry is $p(q) = 14-q$ where $q = q_1 + q_2$.
\begin{parts}
  \part Verify that the optimal level of production for a joint monopoly is $q^M=6$ while the Cournot equilibrium is for each firm to product $q^C = 4$.
  \fullwidth{Suppose that each firm can only choose between two quantities to produce: either $\frac{1}{2}q^M=3$ (half the joint monopoly quantity) or $q^C=4$ (the Cournot quantity). }
  \part Write down the one-shot, simultaneous move game corresponding to this simplified set-up and show that each firm has a dominant strategy. Discuss your results.
\end{parts}

\fullwidth{\section{Microeconomic Theory II}
Attempt these if you have taken or are currently taking \textbf{Advanced Microeconomics}.} 

\question A village has $n$ residents who each obtain utility from private goods and a public good: flood defences. Resident $i$ has utility function $U_i = y_i + a Q - \frac{Q^2}{100}$ where $Q$ is the level of flood defences and $y_i$ is individual $i$'s expenditure on private goods. Flood defences cost $c$ per unit.
\begin{parts}
  \part Find the marginal private benefit of flood defences for resident $i$. Write down and explain the Samuelson condition for optimal provision of the public good.
  \fullwidth{For the rest of this question, assume that the residents agree to provide flood defences through voluntary contributions. Each resident $i$ purchases a quantity $q_i$ of flood defences, taking the contributions of all other residents as given.}
  \part Find the reaction function of person 1 as a function of the contributions of all other residents. 
  \part In a symmetric equilibrium in which each person provides the same amount, how much of the public good will be provided in total? Why doesn't the answer depend on the number of residents?
  \part Discuss your results, illustrating them in a diagram that shows the private and social marginal benefits and costs as a function of $Q$.
\end{parts}

\question Arthur is risk averse, and his income tomorrow depends on which of two random states of the world occurs. The two states are equally likely. His income will be 8 if state A occurs but only 2 if state B occurs.  
\begin{parts}
  \part Draw Arthur's indifference curves in state-contingent income space, and explain how he could be better off if he were able to buy insurance.
  \fullwidth{Norma is risk neutral. Her income will be 3 if state A occurs and 7 if state B occurs. Suppose that Arthur and Norma can only write contract of the form ``Arthur will give Norma amount $x$ if and only if state A occurs, and Norma will give Arthur an amount $y$ if an only if state B occurs''.}
  \part What shape are Norma's indifference curves? From Arthur's point of view, what is the best contract that Norma would find acceptable and why?
  \fullwidth{Now assume that Arthur's utility as a function of his income $m$ is $\ln m$.}
  \part From Norma's point of view, what is the best contract that Arthur would find acceptable?
  \part Use an Edgeworth box to illustrate your results from the preceding two parts. Highlight the set of efficient risk-sharing contracts. 
\end{parts}



\fullwidth{\section{Macroeconomic Theory I}
Attempt these if you have taken or are currently taking \textbf{Intermediate Macroeconomics} or an \emph{advanced} Introduction to Macroeconomics course.}

\question Suppose that consumers live for 2 periods (the present and future). Each consumer has income $y_1$ in the present and $y_2$ in the future, can borrow and save at the real interest rate $r$ and has well-behaved preferences over current and future consumption, $c_1$ and $c_2$.

\begin{parts}
\part Write down the consumer's budget constraint and draw a diagram to illustrate the optimal choice of consumption over the two periods.
\part Explain carefully why, according to this model, changes in the interest rate may have little effect on saving.
\part Discuss how a temporary increase in income, in the present period only, has different effects from an increase in the same size that is expected to be permanent. What would be the effect of a temporary rise in income if there were many periods in the model? What are the implications of your findings for the marginal propensity to consume?
\part Suppose the government levies a tax $T$ on each consumer in the present, invests the proceeds in bonds paying interest at $r$ and returns the amount $T$ plus interest, to the consumer in the future. How will this affect $c_1$ and $c_2$? Would the answer be different if consumers faced borrowing constraints?
\part Compare the policy implications of this model of consumption with those of the Keynesian consumption function.
\end{parts}


\question A decline in foreign demand for U.S. goods: Suppose that the European and Japanese economies succumb to a recession and reduce their demand for U.S. goods for several years. Using the AS/AD framework, explain the macroeconomic consequences of this shock, both immediately and over time.

\fullwidth{\section{Macroeconomic Theory II} Attempt these if you are have taken or are currently taking \textbf{Advanced Macroeconomics}.}

\question Consider the Phillips curve equation from the IS--PC--MR model, namely
\[
  \pi_t = \pi_{t-1} + \alpha(y_t - y_e) + u_t
\]
Suppose that in period $t$ the monetary authority learns of a persistent cost-push shock, i.e.\ it knows as of period $t$ that $u_t = u_{t+1}=1$, but then expects the shock to be zero from periods $t+2$ onward. 
\begin{parts}
  \part Assuming that the economy was in equilibrium at the inflation target before period $t$, show the position of the economy in the IS--PC--MR diagram in period $t$.
  \part Assuming that private agents have adaptive expectations whereas the monetary authority is forward-looking, describe the path followed by the economy from period $t+1$ until the inflation target is restored.
\end{parts}
\question Briefly answer each of the following parts.
\begin{parts}
\item Under what conditions will the real interest rate in a small open economy equal the world real interest rate? Would your answer differ for nominal interest rates?
\item Are government budget deficits always accompanied by current account deficits?
\item Suppose that governments set their fiscal plans only through to the end of the current parliamentary term. Does this invalidate Ricardian Equivalence?
\end{parts}




\fullwidth{\section{Introductory Econometrics}
Attempt these if you have taken or are currently taking \textbf{Introductory Econometrics}.}

\question Briefly answer each part of this question.
\begin{parts}
  \part Explain how measurement error causes attenuation bias in the linear regression model.
  \part Explain and discuss the following claim: ``most regression studies rely on the conditional independence assumption in order to argue that the estimated coefficients represent causal effects.'' 
  \part Consider the following AR(1) time-series model: $y_t = \alpha + \beta y_{t-1} + \epsilon_t$. What econometric problems arise if $\beta = 1$? What if $\beta < 1$ but close to 1?
  \part Suppose we fit an AR(1) model to a time series dataset and obtained 
  \[
    \hat{y}_t = \underset{(2.125)}{5.057} + \underset{(0.022)}{0.947} y_{t-1}
  \]
  where standard errors are given in parentheses. Can you reject the null hypothesis that the coefficient on $y_{t-1}$ equals $1$ at the 5\% significance level?
  \part What problems are caused by structural breaks in time series? How would you test for a structural break?
\end{parts}
\question Let $Q_t$ be quantity, $P_t$ be price, $U_t$ be a ``demand shock'' and $V_t$ be a ``supply shock'' at a particular point in time $t$. Now suppose that the demand and supply functions for oranges are given by
\begin{align*}
  \ln Q_i &= \beta_0 + \beta_1 \ln P_i + U_i & \text{(Demand)}\\
  \ln Q_i &= \gamma_0 + \gamma_1 \ln P_i + V_i & \text{(Supply)}
\end{align*}
  \begin{parts}
    \part Solve for the equilibrium price and quantity in the market for oranges in terms of the supply and demand shocks. What do your results suggest about the correlation between $\ln P_i$ and $U_i$?
    \part Suppose that we observe the price and quantity $(Q_t, P_t)$ in the orange market for a large number of time periods. We decide to estimate the Demand function by running an ordinary least squares (OLS) regression of $\ln Q_t$ on $\ln P_t$. Would you expect our estimate of $\beta_1$ from this procedure to be too high, too low, or about right? Explain briefly.
    \part Suppose instead that we decide to estimate the Demand function using an instrumental variables approach. Below are two proposals for an instrumental variable that we might use. Briefly discuss the pros and cons of each in this setting.
    \begin{itemize}
      \item[(i)] A measure of average rainfall in the region where the oranges are grown.
      \item[(ii)] Income per capita in the region where the oranges are sold.
    \end{itemize}
  \end{parts}

  \question The following table contains the output from a regression of a person's income in adulthood (aged 35) on the income of their parents, measured when the parents were aged 35, and a dummy variable for whether that person lives in a rural area or an urban area.
  \begin{center}
  \begin{tabular}[h]{lcc}
   \hline
   \multicolumn{3}{c}{\textbf{OLS Regression}}\\
   \multicolumn{3}{c}{Dependent variable: \texttt{log(child income)}}\\ \\
   & coefficient & standard error \\
   \hline
   \texttt{log(parent income)} & 0.568 & 0.005 \\
   \texttt{lives in rural area (=1 if rural, =0 if urban)} & -0.091 & 0.003 \\
   \texttt{constant} & 1.94 & 0.017 \\
   \hline
   Number of observations & \multicolumn{2}{c}{3,056}\\ \\
   & Residual & Total \\ 
   \hline
   Sum of Squares & 13.42 & 73.01\\
   \hline
  \end{tabular}
\end{center}
  \vspace{1em}
  \begin{parts}
    \part Compute the R-squared of this regression and interpret it. A researcher claims: ``the larger the R-squared of a regression, the more likely it is that this regression can be given a causal interpretation''. Do you agree? Why or why not? Explain.
  \item Interpret the coefficient on \texttt{log(parental income)}. Compute and interpret an approximate 95\% confidence interval for this coefficient.
  \item Suppose that you wanted to test the null hypothesis that, for a fixed level of parental income, people who live in rural and urban areas earn the same amount on average. You decide to test against the alternative hypothesis that those in rural areas earn \emph{less} on average. Would you reject at the 1\% level? Explain, making sure to detail all the steps involved in carrying out the test.  
  \end{parts}



\end{questions}

\end{document}
