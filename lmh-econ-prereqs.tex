\documentclass[addpoints,12pt,a4paper,landscape]{article}
\usepackage{amsmath, amssymb}
\usepackage[T1]{fontenc}
%\boxedpoints
%\pointsinmargin

\usepackage{datetime2}
\usepackage{bbm}
\usepackage{tikz}
\usepackage{array}
\usepackage{multirow}
\usepackage{amsmath}
\usepackage{amsthm}
\usepackage{amssymb}
\usepackage[left=0.8in,right=0.8in,top=0.8in,bottom=0.8in]{geometry}
\usepackage{enumitem}
\usepackage{array}     
\usepackage{booktabs}
\usepackage{multirow}
\usepackage{array}
\usepackage{hyperref}
\hypersetup{colorlinks=true}

\renewcommand{\thesection}{\Alph{section}}
\renewcommand{\arraystretch}{1.35}


\begin{document}


\begin{center}

  \textsc{\Large Economics Course Prerequisites \\ \medskip \normalsize Lady Margaret Hall Visiting Student Program \\ \vspace{0.5em} Version: \DTMtoday, \DTMcurrenttime}

  \vspace{1ex}

  \href{https://raw.githubusercontent.com/fditraglia/lmh-econ-diagnostic/main/lmh-econ-prereqs.pdf}{\underline{Click here for latest version}}

\vspace{1em}

\end{center}


\begin{center}
    \begin{tabular}{p{0.4cm}p{5cm}p{18cm}}
      \multicolumn{2}{p{5.5cm}}{\textbf{LMH/Oxford Prerequisite}}   &  \textbf{Notes on US equivalents}\\ \hline

A	& Calculus for Economics	& Any calculus course that covers basic integration, partial differentiation, constrained optimisation and the implicit function theorem (informally).\\

B	& \raggedright Introductory Probability \& Statistics	& A one-semester course, possibly called \emph{Statistics for Economics} or \emph{Statistics for Economics and Business} that satisfies the home university's prerequisite for \emph{Econometrics}.\\

C	& Econometrics	& \emph{Introductory Econometrics}, \emph{Econometrics} or \emph{Applied Econometrics} depending on the university.\\

D	& Micro Theory I	& \emph{Intermediate Micro (with calculus)} OR \emph{Intermediate Micro} PLUS \emph{Calculus for Economics}. At some universities (but very few)  \emph{Principles of Microeconomics} may reach the required level.  Students should submit a detailed syllabus for review if they wish their Principles course, or another course, to be considered.\\

E	& Micro Theory II	& \emph{Advanced Microeconomics}. At some universities (but very few)  \emph{Intermediate Microeconomics} or \emph{Microeconomic Theory} may reach the required level.  Students should submit a detailed syllabus for review if they wish their Intermediate course, or another course, to be considered. \\

F	& \raggedright Introduction to Game Theory	&  Any \emph{Microeconomic Theory} or \emph{Game Theory} course that covers non-cooperative game theory with discrete and continuous actions, simultaneous, sequential and repeated games and solution concepts including dominance, Nash equilibrium and subgame perfect equilibrium. \\

G	& Macro Theory I	& \emph{Intermediate Macro (with calculus)} OR \emph{Intermediate Macro} PLUS \emph{Calculus for Economics}. At some universities (but very few) \emph{Principles of Macroeconomics} may reach the required level.  Students should submit a detailed syllabus for review if they wish their Principles course, or another course, to be considered.\\

H	&  Macro Theory II	& \emph{Advanced Macroeconomics}. 
At some universities (but very few)  \emph{Intermediate Macroeconomics} or \emph{Macroeconomic Theory} may reach the required level.  Students should submit a detailed syllabus for review if they wish their Intermediate course, or another course, to be considered.

\end{tabular}
\end{center}

\end{document}